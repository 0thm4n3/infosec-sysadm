\section{Scope and aims}
\label{sec:aim}
The course treats information security from a user, organization and technical 
perspective.
The first part of the course concerns security on a strategic level, i.e.\ 
managing security within an organization.
The second part of the course focuses on the operative parts, i.e.\ security 
mechanisms and principles for design of secure systems.
In full, the course aims at giving you an understanding of threats to security 
and how to work to protect against these.

\subsection{\acpl*{ILO}}

More concretely, the \acp{ILO} of the course are the following.
After completing the course, you should be able to:
\begin{itemize}
  \item \emph{evaluate} the usability of security solutions and \emph{suggest} 
    improvements that improve usability and security.
  \item \emph{evaluate} threats, possible protection mechanisms and 
    \emph{design} a high-level approach to protection which considers 
    usability.
  \item \emph{overview} the field of information security, \emph{understand} 
    your own limits and where to search for solutions, e.g\ experts or 
    published research results that are relevant to the solution of a problem.
  \item \emph{analyse and apply} the results of published research in the 
    security field.
  \item \emph{apply} the Swedish Civil Contingency Agency's Framework for 
    Information Security Management Systems (ISO 27000) to analyse, assess and 
    improve the information security in an organization.
\end{itemize}
The course has a variety of learning sessions designed to ensure that you learn 
these \acp{ILO}.
Each such session has a set of further specified \acp{ILO} that will help you 
achieve the \acp{ILO} above.

\subsection{Prerequisites}

\dots


\section{Course structure and content overview}
\label{sec:outline}

The first part of the course covers information security on a strategic level, 
this concerns organizational management systems for information security: how 
to implement these and how to continuously run them in an organization.
The main material used for this part~\cite{%
  MSB2011itm,MSB2011sle,MSB2011p,%
	MSB2011v,MSB2011r,MSB2011gap,MSB2011gb,%
	MSB2011vs,MSB2011us,MSB2011upo,%
	MSB2011pg,MSB2011koa,MSB2011i,%
	MSB2011o,MSB2011g,MSB2011lg,%
	MSB2011ulo,MSB2011kf,MSB2011fa%
} is produced by the Swedish Civil Contingencies Agency (MSB) and is based on 
the ISO 27000 standard documents.

The second part of the course will focus on the content of Anderson's book 
\citetitle{Anderson2008sea}~\cite{Anderson2008sea} and Gollmann's book 
\citetitle{Gollmann2011cs}~\cite{Gollmann2011cs}.
The focus in the second part of the course is on security mechanisms and how to 
use these in secure systems.
There is also some additional material for this part of the course, e.g.\ 
research papers and some other material.

\subsection{Teaching and tutoring}

The course is taught using lectures, seminars, laboratory assignments and, 
finally, a project.
All assignments are numbered consecutively prefixed with an \enquote{L} for 
laboratory assignments, \enquote{S} for seminar assignments and \enquote{M} for 
memos.

\subsection{Schedule}

You will find an outline for a schedule for the course in \cref{Schedule}.
You are free to follow this schedule or any schedule you make for yourself, but 
the learning and tutoring sessions, deadlines etc.\ will follow this schedule.
The detailed reading instructions for each item in the schedule can be found in 
the following sections.

\begin{table}
	\centering
  \caption{%
    A summary of the parts of the course and when they will (or should) be 
    done.
    The table is adapted to taking this course on half-time study rate.
  }\label{Schedule}
  \begin{tabular}{lp{9cm}}
    \toprule
    \textbf{Course Week}	& \textbf{Work} \\
    \midrule
    1
      & Course Start/Foundations of Security\\
      & Lecture on Security Usability\\
      %& Start working on L0 (privacy)\\
      & Lecture on MSB's Framework, Part I\\
      & Start working on M1 (isms)\\
      & Lecture on MSB's Framework, Part II\\
      & Start working on M2, prepare S3 (risk)\\
    \midrule
    2
      & Lecture on Information Theory\\
      & Lecture on Cryptographic Mechanisms, Part I\\
      & Lecture on Cryptographic Mechanisms, Part II\\
      & First grading of M1 (isms), M2 (risk)\\
    \midrule
    3
      & Lecture on Identification and Authentication\\
      & First seminar session S3 (risk)\\
    \midrule
    4
      & Lecture on Access Control\\
      & Lecture on Secure Protocols\\
      & Lecture on Accountability and Non-Repudiation\\
      & Lab session L4 (pwdguess)\\
      & Lab session L6 (pricomlab)\\
      & First seminar session S5 (pwdpolicies)\\
    \midrule
    5
      & Lecture on Software Security\\
      & Lecture on Trusted Computing\\
      & Lecture on Side-Channels\\
      & Lab session L4 (pwdguess)\\
      & Lab session L6 (pricomlab)\\
    \midrule
    6
      & Tutoring session for project\\
      & Lab session L4 (pwdguess)\\
      & Lab session L6 (pricomlab)\\
    \midrule
    7
      & Tutoring session for project\\
      & Presentation for S7 (review)\\
      & Lab session L4 (pwdguess)\\
      & Lab session L6 (pricomlab)\\
    \midrule
    8
      %& Presentation for L0 (privacy)\\
      & Tutoring session for project\\
    \midrule
    9
      & Tutoring session for project\\
    \midrule
    10
      & Presentation P8 (research)\\
      & Second grading of M1 (isms), M2 (risk)\\
      & Second seminar session for S3 (risk), S5 (pwdpolicies), S7 (review)\\
      & Final lab session L4 (pwdguess), L6 (pricomlab)\\
    \midrule
    +3 months
      & Second presentation P8 (research)\\
      & Final grading of M1 (isms), M2 (risk)\\
      & Final seminar session for S3 (risk), S5 (pwdpolicies), S7 (review)\\
    \midrule
    +6 months
      & Final presentation P8 (research)\\
    \bottomrule
  \end{tabular}
\end{table}


\section{Course content}

This section summarizes the material covered by the lectures and assignments, 
i.e.~what you should read for each of them.
It is divided by topics and ordered according to progression of the course.

\subsection{Foundations of Security}
\input{foundations-abstract.tex}

%\subsection{L0 Privacy is Dead}
%\input{privacy-abstract.tex}
%
\subsection{MSB's Framework}
\input{msbframework-abstract.tex}

\subsection{M1 Information Security Management System}
\input{ismsmemo-abstract.tex}

\subsection{M2 and S3 Assessment and Risk Analysis}
\input{risksem-abstract.tex}

\subsection{Information Theory}
\input{infotheory-abstract.tex}

\subsection{Cryptographic Mechanisms}
\input{crypto-abstract.tex}

\subsection{Identification and Authentication}
\input{auth-abstract.tex}

\subsection{Security Usability}
\input{usability-abstract.tex}

\subsection{Access Control}
\input{accessctrl-abstract.tex}

%\subsection{Multi-Level and Multi-Lateral Security}
%\input{lvlltrl-abstract.tex}

\subsection{Secure Protocols}
\input{proto-abstract.tex}

\subsection{L4 Password Cracking and Social Engineering}
\input{pwdguess-abstract.tex}

\subsection{S5 Password Policies}
\input{pwdpolicies-abstract.tex}

\subsection{Accountability and Non-Repudiation}
\input{accountability-abstract.tex}

\subsection{L6 Privacy of Communication}
\input{pricomlab-abstract.tex}

\subsection{Software Security}
\input{software-abstract.tex}

\subsection{DRM and Trusted Computing}
\input{trustcomp-abstract.tex}

\subsection{Side-Channels}
\input{sidechannels-abstract.tex}

\subsection{P6 Integrating security and usability in development}
\input{project-abstract.tex}


\section{Assessment}
\label{Assessment}

This section explains how the course modules are graded and mapped to LADOK\@.
\Cref{LADOKTable} visualizes the relations between modules, credits, grades and 
LADOK\@.

\begin{table}
  \centering
  \caption{%
    Table summarizing course modules and their mapping to LADOK\@.
    P means pass, F means fail.
    A--E are also passing grades, where A is the best.
  }\label{LADOKTable}
  \begin{tabular}{rrll}
    \toprule
    LADOK & Credits (ECTS)  & Grade       & Course Assignments\\
    \midrule
    I104  & 1.5             & P, F        & M1, M2, S3, S5\\
    L104  & 1.5             & P, F        & L4, L6\\
    R104  & 4.5             & A--F        & P8 (and S7)\\
    \midrule
    Total & 7.5             & A--F        & P8\\
    \bottomrule
  \end{tabular}
\end{table}

The project report is graded from A to F, where A--E are for passing and F and 
Fx are for failing.
The project also includes an oral presentation and a seminar (S7).
These are both graded pass (P) or fail (F), and are reported with the project 
to LADOK\@.
The grade of the project will also be the grade of the course total.

\subsection{Handed-In Assignments}

In general, all hand-ins in the course must be in a \enquote{passable} 
condition; i.e.~they must be well-written, grammatically correct and without 
spelling errors, have citations and references according to~\cite{IEEEcitation} 
(see also~\cite{PurdueCitation} for a tutorial), and finally fulfil all 
requirements from the assignment instruction.
If you hand something in which is not in this condition, you will receive an 
F without further comment.

All material handed-in must be created by yourself, or, in the case of group 
assignments, created by you or one of the group members.
When you refer to or quote other texts, then you must provide a correct list of 
references and, in the case of quotations, the quoted text must be clearly 
marked as quoted.
If any part of the document is plagiarised you risk being suspended from study 
for a predetermined time, not exceeding six months, due to disciplinary 
offence.
If it is a group assignment, all group members will be held accountable for 
disciplinary offence unless it is clearly marked in the work who is responsible 
for the part containing the plagiarism.

If cooperation takes place without the assignment instruction explicitly 
allowing this, this will be regarded as a disciplinary offence with the risk of
being suspended for a predetermined time, not exceeding six months.
Unless otherwise stated, all assignments are to be done individually.

\subsection{\enquote{What if I'm not done in time?}}
\label{sec:late}
%\subsection{\enquote{What if I'm not done in time?}}

The deadlines on this course are of great importance, make sure to keep these!
%You must have completed the introductory assignment within its deadline.
%If you do not do this you will be deregistered from the course and your place 
%will be open to other students.

For seminars and presentations there will be three sessions during the course 
of a year, if you cannot make it to any of those you will have to return the 
next time the course is given; i.e.~up to a year later.
All of these sessions will be in the course schedule (in the Student Portal).
If you miss a deadline for the preparation for a seminar session, then you have 
to go for the next seminar even if the first seminar has not passed yet.

Written assignments are graded once during the course, most often shortly after 
the deadline of the assignment.
After the course you are offered two more attempts within a year.
In total you have three chances for having your assignments graded over the 
period of a year.
After that you should come back the next time the course is given.

No tutoring is planned after the end of the course, i.e.~after the last 
tutoring session scheduled in the course schedule.
If you are not done with your assignments during the course and want to be 
guaranteed tutoring you have to reregister for the next time the course is 
given.
Reregistration is a lower priority class of applicants for a course, all 
students applying for the course the first time have higher priority -- this 
includes reserves too.

%If you by the end of the course have a majority of the assignments left undone 
%you will have to reregister for the course the next time it is given.
%Whether you have completed the majority of the assignments or not is up to the 
%teacher to decide.
%Talk to the teacher to see if you have to reregister or can just hand in the 
%missing assignments.

Thus, if you feel that you will not be done with the course on time, it is 
better to stop the course at an early stage.
If you register a break within three weeks of the course start, you will be in 
the higher priority class of applicants the next time you apply for the course.
You can register such a break yourself in the Student Portal.




\printbibliography{}
