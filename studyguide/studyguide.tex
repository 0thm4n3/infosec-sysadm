\documentclass[a4paper]{llncs}
\usepackage[utf8]{inputenc}
\usepackage[swedish,english]{babel}
\usepackage{varioref}
\usepackage[hyphens]{url}
\usepackage{hyperref}
\usepackage{cleveref}
\usepackage{booktabs}
\usepackage[binary,squaren]{SIunits}
\usepackage[strict]{csquotes}
\usepackage{acro}

\usepackage[std]{libbib}

\usepackage[natbib,style=numeric-comp,maxbibnames=99,sorting=none]{biblatex}
\addbibresource{pricomlab.bib}
\addbibresource{pwdguess.bib}
\addbibresource{pwdpolicies.bib}
\addbibresource{ismsmemo.bib}
\addbibresource{risksem.bib}
\addbibresource{msbintro.bib}
\addbibresource{msbforts.bib}
\addbibresource{software.bib}
\addbibresource{usability.bib}
\addbibresource{infotheory.bib}
\addbibresource{trustcomp.bib}
\addbibresource{fverif.bib}
\addbibresource{sidechannels.bib}
\addbibresource{auth.bib}
\addbibresource{ac.bib}
\addbibresource{accountability.bib}
\addbibresource{foundations.bib}

\pagestyle{plain}

\title{%
  The Complete Study Guide for\\
  DV026G Information Security
}
\author{%
  Daniel Bosk\inst{1,2}
}
\institute{%
  Department of Information and Communication Systems\\
  Mid Sweden University, Sundsvall\\
  \and
  School of Computer Science and Communication\\
  KTH Royal Institute of Technology, Stockholm\\
}
\date{\today}

\begin{document}
\maketitle


\section{Scope and Aims}
\label{sec:aim}
The course treats information security from a user, organization and 
technological perspective.
The first part of the course concerns security on a strategic level, 
i.e.~working with security in general within an organization.
The second part of the course focuses on the operative parts, i.e.~security 
mechanisms and principles for design of secure systems.
In full, the course aims at giving you an understanding for threats to security 
and how to work to protect against these.

A more concrete summary of the course achievements are the following, after 
completing the course you should be able to:
\begin{itemize}
%  \input{privacy-aim.tex}
%  \input{ismsmemo-aim.tex}
%  \input{risksem-aim.tex}
%  \input{pwdguess-aim.tex}
%  \input{pwdpolicies-aim.tex}
%  \input{review-aim.tex}
%  \input{project-aim.tex}
  \item Explain basic concepts and models in information security.
  \item Analyse threats and possible protection mechanisms.
  \item Apply the Swedish Civil Contingency Agency's Framework for Information 
    Security Management Systems to analyse, assess and improve the information 
    security in an organization.
\end{itemize}


\section{Overview of Structure and Content}
\label{sec:outline}
% Hot mot informationssäkerheten: exempelvis sabotageprogram, social 
% engineering, avlyssning och förvanskning av data.
%
% Skydd för informationssäkerheten: exempelvis behörighetskontroll, kryptering, 
% digitala signaturer och autenticering, modeller för säkerhet.
%
% Samverkan mellan människor, teknik och organisationer ur ett 
% säkerhetsperspektiv: exempelvis tillämpbara lagar och förordningar, 
% säkerhetsmedvetande hos användare, säkerhetspolicyer och MSB:s ramverk för 
% informationssäkerhet.

The first part of the course, the one covering information security on 
a strategic level, concerns organizational management systems for information 
security; how to implement these and how to continuously run them in an 
organization.
The main material used for this part~\cite{%
  MSB2011itm,MSB2011sle,MSB2011p,%
	MSB2011v,MSB2011r,MSB2011gap,MSB2011gb,%
	MSB2011vs,MSB2011us,MSB2011upo,%
	MSB2011pg,MSB2011koa,MSB2011i,%
	MSB2011o,MSB2011g,MSB2011lg,%
	MSB2011ulo,MSB2011kf,MSB2011fa%
} is produced by the Swedish Civil Contingencies Agency (MSB) and is based on 
the ISO 27000 standard documents.

The second part of the course will focus on the content of Anderson's book 
\citetitle{Anderson2008sea}~\cite{Anderson2008sea}.
Gollmann's book \citetitle{Gollmann2011cs}~\cite{Gollmann2011cs} is also 
useful, it has more technical details than~\cite{Anderson2008sea}.
Although references to~\cite{Gollmann2011cs} is provided in the reading 
instructions, those are not necessary for this course.

The focus of the second part is on actual attacks, security mechanisms, and how 
to use these in secure protocols.
There are also some additional material for this part of the course, 
e.g.~research papers~\cite{%
  Juels2012sha,%
  Komanduri2011opa%
} and some other material~\cite{Bosk2013gl,Bosk2013itn}.
In addition to these there will also be some news articles~\cite{%
  Honan2012haa,%
  Zetter2012hnt,%
  Fisher2011rsa,%
  Hunt2011abs,%
  Cluley2012twp,%
  Oberheide2010bao,%
  Cubrilovic2009rhf%
} which has documented some of the major security incidents during the past few 
years.
MSB also has the website CERT-SE~\cite{CERT-SE} which has some interesting 
references and security news, e.g.~virus epidemics in Sweden.

\subsection{Teaching}

The course is taught using lectures, individual laboratory assignments, 
seminars, a project, and finally an exam.
You can find a more detailed timetable, containing lab sessions etc., in the 
following section.
All assignments are numbered consecutively prefixed with an \enquote{L} for 
laboratory assignments, \enquote{S} for a seminar assignment, and \enquote{M} 
for memos.
For details on the examination of these and more information about deadlines, 
see \cref{Examination}.

\subsection{Course Schedule}
\label{sec:schedule}
To make your reading of the course easier you are presented with a suggested 
schedule in this section.
You are free to follow this schedule or any schedule you make for yourself, but 
the deadlines, laboratory sessions, and lectures will follow this schedule.
You will find a short summary of schedule in \cref{Schedule}.
The detailed reading instructions for each item in the schedule can be found in 
the following sections.

\begin{table}
	\centering
  \caption{%
    A summary of the parts of the course and when they will (or should) be 
    done.
    The table is adapted to taking this course on half-time study rate.
  }\label{Schedule}
  \begin{tabular}{rp{9cm}}
    \textbf{Course Week}	& \textbf{Work} \\
    \toprule
    1
      & Course Start/Foundations of Security\\
      & Lecture on MSB's Framework, Part I\\
      & Start working on M1 (isms)\\
      & Lecture on MSB's Framework, Part II\\
      & Start working on M2, prepare S3 (risk)\\
    \midrule
    2
      & Lecture on Information Theory\\
      & Lecture on Cryptographic Mechanisms, Part I\\
      & Lecture on Cryptographic Mechanisms, Part II\\
      & First grading of M1 (isms), M2 (risk)\\
    \midrule
    3
      & Lecture on Identification and Authentication\\
      & Lecture on Security Usability\\
      & Seminar session S3 (risk)\\
    \midrule
    4
      & Lecture on Access Control\\
      & Lecture on Secure Protocols\\
      & Lecture on Accountability and Non-Repudiation\\
      & Seminar session S5 (pwdpolicies)\\
      & Lab session L4 (passwd), L6 (privcomm)\\
    \midrule
    5
      & Lecture on Software Security\\
      & Lecture on DRM and Trusted Computing\\
      & Lecture on Side-Channels\\
      & Lab session L4 (passwd), L6 (privcomm)\\
    \midrule
    6
      & Tutoring session for project, L4 (passwd), L6 (privcomm)\\
      & First grading of L4 (passwd), L6 (privcomm)\\
    \midrule
    7
      & Tutoring session for project, L4 (passwd), L6 (privcomm)\\
    \midrule
    8
      & Tutoring session for project, L4 (passwd), L6 (privcomm)\\
    \midrule
    9
      & Tutoring session for project, L4 (passwd), L6 (privcomm)\\
    \midrule
    10
      & First exam\\
      & First grading of project\\
      & Second grading of M1 (isms), M2 (risk)\\
      & Second seminar session for S3 (risk), S5 (pwdpolicies)\\
      & Second grading of L4 (passwd), L6 (privcomm)\\
    \midrule
    +3 months
      & Second exam\\
      & Second grading of project\\
      & Final grading of M1 (isms), M2 (risk)\\
      & Final seminar session for S3 (risk), S5 (pwdpolicies)\\
      & Final grading of L4 (passwd), L6 (privcomm)\\
    \midrule
    +6 months
      & Final exam\\
      & Final grading of project\\
    \bottomrule
  \end{tabular}
\end{table}


\section{Course Content}

This section summarizes the material covered by the lectures and assignments, 
i.e.~what you should read for each of them.
It is divided by topics and ordered according to progression of the course.

\subsection{Foundations of Security}
\input{foundations-lit.tex}

\subsection{MSB's Framework, Part I}
\input{msbintro-lit.tex}

\subsection{M1 Information Security Management System}
\input{ismsmemo-lit.tex}

\subsection{MSB's Framework, Part II}
\input{msbforts-lit.tex}

\subsection{M2 and S3 Assessment and Risk Analysis}
\input{risksem-lit.tex}

\subsection{Information Theory}
\input{infotheory-lit.tex}

\subsection{Cryptographic Mechanisms}
\input{crypto-lit.tex}

\subsection{Identification and Authentication}
\input{auth-lit.tex}

\subsection{Security Usability}
\input{usability-lit.tex}

\subsection{Access Control}
\input{accessctrl-lit.tex}

%\subsection{Multi-Level and Multi-Lateral Security}
%\input{lvlltrl-lit.tex}

\subsection{Secure Protocols}
\input{proto-lit.tex}

\subsection{L4 Password Cracking and Social Engineering}
\input{pwdguess-lit.tex}

\subsection{S5 Password Policies}
\input{pwdpolicies-lit.tex}

\subsection{Accountability and Non-Repudiation}
\input{accountability-lit.tex}

\subsection{L6 Privacy of Communication}
\input{pricomlab-lit.tex}

\subsection{Software Security}
\input{software-lit.tex}

\subsection{DRM and Trusted Computing}
\input{trustcomp-lit.tex}

\subsection{Side-Channels}
\input{sidechannels-lit.tex}

\subsection{P7 A Short Gap Analysis}

The project consists of doing a short gap analysis.
As such you must have read MSB's material~\cite{%
  MSB2011itm,MSB2011sle,MSB2011p,MSB2011v,MSB2011r,%
  MSB2011gap,MSB2011vs,MSB2011us,MSB2011upo,%
	MSB2011pg,MSB2011koa,MSB2011i,MSB2011o,MSB2011g,%
	MSB2011lg,MSB2011ulo,MSB2011kf,MSB2011fa%
}.


\section{Examination}
\label{Examination}

This section explains how the course modules are graded and mapped to LADOK\@.
\Cref{LADOKTable} visualizes the relations between modules, credits, grades and 
LADOK\@.

\begin{table}
  \centering
  \caption{%
    Table summarizing course modules and their mapping to LADOK\@.
    P means pass, F means fail.
    A--E are also passing grades, where A is the best.
  }\label{LADOKTable}
  \begin{tabular}{rrll}
    LADOK & Credits (ECTS)  & Grade       & Course Assignments\\
    \toprule
    X104  & 0.0             & P, F        & M1\\
    I104  & 1.5             & P, F        & M2, S3, S5\\
    L104  & 1.5             & P, F        & L4, L6\\
    P104  & 3.0             & A, C, E, F  & P7\\
    T104  & 1.5             & A, C, E, F  & Exam\\
    \midrule
    Total & 7.5             & A--F        & Average of P7 and exam\\
    \bottomrule
  \end{tabular}
\end{table}

The project report is graded from A, C, or E for for passing or F or Fx for 
failing.
The exam is also graded A, C, or E for passing or F or Fx for failing.
The final grade for the course is the average of the exam and the project.
For example, if you have an A on the exam but a C on the project, then you will 
get a B as the course total.

\subsection{Handed-In Assignments}

In general, all hand-ins in the course must be in a \enquote{passable} 
condition; i.e.~they must be well-written, grammatically correct and without 
spelling errors, have citations and references according to~\cite{IEEEcitation} 
(see also~\cite{PurdueCitation} for a tutorial), and finally fulfil all 
requirements from the assignment instruction.
If you hand something in which is not in this condition, you will receive an 
F without further comment.

All material handed-in must be created by yourself, or, in the case of group 
assignments, created by you or one of the group members.
When you refer to or quote other texts, then you must provide a correct list of 
references and, in the case of quotations, the quoted text must be clearly 
marked as quoted.
If any part of the document is plagiarised you risk being suspended from study 
for a predetermined time, not exceeding six months, due to disciplinary 
offence.
If it is a group assignment, all group members will be held accountable for 
disciplinary offence unless it is clearly marked in the work who is responsible 
for the part containing the plagiarism.

If cooperation takes place without the assignment instruction explicitly 
allowing this, this will be regarded as a disciplinary offence with the risk of
being suspended for a predetermined time, not exceeding six months.
Unless otherwise stated, all assignments are to be done individually.

\subsection{\enquote{What if I'm not done in time?}}
\label{sec:late}
The deadlines on this course are of great importance, make sure to keep these!
%You must have completed the introductory assignment within its deadline.
%If you do not do this you will be deregistered from the course and your place 
%will be open to other students.

For seminars and presentations there will be three sessions during the course 
of a year, if you cannot make it to any of those you will have to return the 
next time the course is given; i.e.~up to a year later.
All of these sessions will be in the course schedule (in the Student Portal).
If you miss a deadline for the preparation for a seminar session, then you have 
to go for the next seminar even if the first seminar has not passed yet.

Written assignments are graded once during the course, most often shortly after 
the deadline of the assignment.
After the course you are offered two more attempts within a year.
In total you have three chances for having your assignments graded over the 
period of a year.
After that you should come back the next time the course is given.

No tutoring is planned after the end of the course, i.e.~after the last 
tutoring session scheduled in the course schedule.
If you are not done with your assignments during the course and want to be 
guaranteed tutoring you have to reregister for the next time the course is 
given.
Reregistration is a lower priority class of applicants for a course, all 
students applying for the course the first time have higher priority -- this 
includes reserves too.

%If you by the end of the course have a majority of the assignments left undone 
%you will have to reregister for the course the next time it is given.
%Whether you have completed the majority of the assignments or not is up to the 
%teacher to decide.
%Talk to the teacher to see if you have to reregister or can just hand in the 
%missing assignments.

Thus, if you feel that you will not be done with the course on time, it is 
better to stop the course at an early stage.
If you register a break within three weeks of the course start, you will be in 
the higher priority class of applicants the next time you apply for the course.
You can register such a break yourself in the Student Portal.


\printbibliography{}
\end{document}
