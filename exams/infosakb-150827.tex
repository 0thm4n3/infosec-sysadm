% $Id$
% Author:  Daniel Bosk <daniel.bosk@miun.se>
\documentclass[svv,addpoints]{miunexam}
\usepackage[utf8]{inputenc}
\usepackage[T1]{fontenc}
\usepackage[swedish,english]{babel}
\usepackage[hyphens]{url}
\usepackage{hyperref}
\usepackage{color}
\usepackage{prettyref,varioref}
\usepackage{subfigure}
\usepackage{amsmath,amssymb}
\usepackage{listings}
\usepackage{authblk}

\usepackage[autopunct]{csquotes}
\MakeBlockQuote{<}{|}{>}
\EnableQuotes{}

\usepackage[natbib,style=alphabetic,maxbibnames=99,backend=bibtexu]{biblatex}
\addbibresource{literature.bib}

\usepackage[varioref,prettyref,listings]{miunmisc}

%\printanswers{}

\examtype{Final exam}
\courseid{DV026G}
\course{Information Security}
\date{2015-08-27}
\author{%
  Daniel Bosk
}
\affil{%
  Department of Information and Communication Systems,\\
  Mid Sweden University, SE-851\,70 Sundsvall\\
  Email: \href{mailto:daniel.bosk@miun.se}{daniel.bosk@miun.se}\\
  Phone: 010-142\,8709
}

\DeclareMathOperator{\hmac}{HMAC}
\DeclareMathOperator{\xor}{\oplus}
\DeclareMathOperator{\concat}{||}

\begin{document}
\maketitle
\thispagestyle{foot}

\section*{Instructions}
\label{sec:Instructions}
Carefully read the questions before you start answering them.
Note the time limit of the exam and plan your answers accordingly.
Only answer the question, do not write about subjects remotely related to the
question.

Write your answers on separate sheets, not on the exam paper.
Only write on one side of the sheets.
Start each question on a new sheet.
Do not forget to \emph{motivate your answers.}

Make sure you write your answers clearly, if I cannot read an answer the answer
will be awarded no points---even if the answer is correct.
The questions are \emph{not} sorted by difficulty.

\begin{description}
  \item[Time] 5 hours.
  \item[Aids] Dictionary, course material and notes.
  \item[Maximum points] \numpoints{}
  \item[Questions] \numquestions{}
\end{description}

\subsection*{Preliminary grades}
The following grading criteria applies:
E \(\geq 50\%\),
D \(\geq 60\%\),
C \(\geq 70\%\), 
B \(\geq 80\%\),
A \(\geq 90\%\).
No question must be awarded zero points.


\clearpage
\section*{Questions}
The questions are given below.
They are not given in any particular order.

\begin{questions}
\question\label{q:accountability:E:C}
  Separation of duties is a core concept for security.
  \begin{parts}
    \part[2] Describe the two types of separation of duties.
    \part[1] What is the main reason for separation of duties?
  \end{parts}

  \begin{solution}
    There are two types of separation of duties:
    dual control and functional separation.
    Dual control means that two or more subjects must act together (at the same 
    time) to authorize a transaction.
    Functional separation means that several functions are needed to authorize 
    a transaction---e.g.~create a transaction and verify it---and one subject 
    is not allowed to do both functions.

    The reason for separation of duties to make it impossible for one malicious 
    subject to compromise a system.
    With separation of duties the malicious subject must persuade one or more 
    other subjects to collude.
  \end{solution}


  
\question\label{q:passwd:auth:E}
  Explain the following terms:
  \begin{parts}
    \part[1] Brute force
    \part[1] Dictionary attack
    \part[1] Hash table
    \part[1] Social engineering
    \part[1] Two-factor authentication
    \part[1] Phishing
  \end{parts}


  
\question\label{q:crypto:foundations:E}
  Explain the following terms:
  \begin{parts}
    \part[1] Confidentiality
    \part[1] Integrity
    \part[1] Availability
    \part[1] Accountability
    \part[1] Non-Repudiation
  \end{parts}

  \begin{solution}
    See~\cite{Gollmann2011cs} and~\cite{Anderson2008sea} for definitions.
  \end{solution}


  
\question[5]\label{q:msb:E:C}
  Your boss is finally convinced that the company needs an Information Security 
  Management System (ISMS, Swe.~`ledningssystem för informationssäkerhet').
  He comes to ask you how an ISMS is best implemented, explain how that is 
  done.


  
\question\label{q:trustcomp:C:A}
  The company you work for want to implement extra features as in-app purchases 
  for the company's main product.
  You are currently in a meeting about that particular topic, the chairperson 
  of the meeting points at you and asks:
  \enquote{%
    How would you design this system?
    The customers must pay for the features, for every installation, we cannot 
    allow them to just buy once and copy later.
    Give us an overview.
  }
  \begin{parts}
    \part[2] Outline the main points in your strategy.
    \part[2] There are some things you simply cannot protect against.
    Explain the limits of systems such as these, so that everyone present 
    understands the limits.
  \end{parts}

  \begin{solution}
    If possible, put the features on a server and execute them remotely.
    This requires proper authentication.
    Ensure this authentication cannot be broken, so that two devices can use 
    the same credentials.
    Try to use a platform where a user must \enquote{root} their phones to 
    violate this.

    The limitations of these systems are that they try to protect code running 
    in a hostile environment.
    If the user has total control of the hardware, then the user can simply 
    copy the software to another device.
    The user can even make arbitrary changes to the running code.
    This means that any keys can be copied from one device to another.
    Even if the app is programmed to transmit the device ID, the app can be 
    reprogrammed to transmit a hard-coded static device ID\@.
  \end{solution}


\question\label{q:auth:E:C}
  Describe the terms
  \begin{parts}
    \part[2] identification and
    \part[2] authentication.
  \end{parts}
  Make sure to illustrate your explanations by examples.
  You must also give an example of a mechanism for each of the terms.

  \begin{solution}
    In identification you claim an identity.
    This can be done using e.g.~a username, fingerprint or DNA sequence.

    In authentication you prove you are who you claim you are.
    This can be done using e.g.~\emph{who} you are (biometric), \emph{where} 
    you are (location) or what you \emph{do} (biometric), something you 
    \emph{have} (e.g.~BankID), or something you \emph{know} (password).
  \end{solution}


  
\question\label{q:trustcomp:crypto:E:C}
  A user wishes to provide confidentiality to a file.
  \begin{parts}
    \part[3] She can accomplish this through mechanisms provided in the 
    operating system.
    Explain how this works and what are the limits.

    \part[3] She can also accomplish this through purely cryptographic 
    mechanisms.
    Explain how this works and what are the limits.
  \end{parts}

  \begin{solution}
    The first way she's securing her file is by using access control mechanisms 
    in the operating system (OS).

    Assuming we have physical access to the computer, then we can just read the 
    raw data from the disk.
    This can be accomplished by either booting our own OS on her computer, or 
    by removing the disk.

    If we don't have physical access we can always try to bypass the access 
    control mechanisms in other ways, e.g.\ by gaining privileges in the system 
    or seeing if the OS has failed to protect reading from the raw disk (i.e.\ 
    not using the file system).

    The main point here is that the operating system must be working correctly 
    for its mechanisms to be effective.
    The \emph{running} operating system will provide confidentiality by not 
    allowing other users' requests to open the file.

    The most obvious way to have system independent security for this file is 
    to encrypt it, i.e.~using cryptographic mechanisms.
    This way no one can read it unless they have access to the key, and this is 
    true no matter if you change the environment.
    (Of course, if the system is untrusted someone can get to the key that way, 
    but that's outside the scope of this question.)
  \end{solution}


  
\question\label{q:accessctrl:trustcomp:E:C:A}
  You have implemented a Mandatory Access Control (MAC) system in the 
  organization you work for.
  \begin{parts}
    \part[4] Explain what properties you might have wanted which made you do 
    this.

    \part[2] Some employees use laptops.
    If you only rely upon the operating system to enforce the policy through 
    MAC, then those employees with laptops might be able to break your policy.
    Give an example of how a user can break your security policy in this 
    situation.
    Make clear what part of your policy is violated and why it can be done.
  \end{parts}

  \begin{solution}
    The most obvious property is confidentiality.
    If we add MAC, then we lower the risk of accidentally exposing confidential 
    data since the system would automatically assign the correct classification 
    to the result.
    (This prevents data in high from going to low.)

    Another property which is plausible is integrity.
    This way the system ensures that we don't trust data more than its source.
    E.g.\ if we download a file from the network, then we shouldn't trust it 
    more than we trust the network.
    This means that the system can e.g.\ disallow executing such a file.
    (We only allow execution of trusted data.)

    A user with a laptop has access to the hardware.
    This means that the user can bypass the operating system.
    By doing that the user can read and write arbitrary data from and to the 
    hard-disk.
  \end{solution}


  
\question\label{q:foundations:E}
  Define the following terms:
  \begin{parts}
    \part[1] Trusted
    \part[1] Trustworthy
    \part[1] Secrecy
    \part[1] Confidentiality
    \part[1] Privacy
    \part[1] Integrity
    \part[1] Authenticity
  \end{parts}

  \begin{solution}
    \citet[ss.\ 13--14]{Anderson2008sea} definierar begreppen enligt följande:
    \begin{description}
      \item[Pålitlighet] Ett system eller principal som innehar pålitlighet 
        (\foreignlanguage{english}{is trusted}) är ett system eller principal 
        som kan bryta din säkerhetspolicy.

      \item[Pålitlig] Ett system eller principal som är pålitlig 
        (\foreignlanguage{english}{is trustworthy}) är ett system eller 
        principal som inte kommer att misslyckas.
        (Den kommer alltså inte att bryta din säkerhetspolicy.)

        Ett exempel för att illustrera skillnaden ges av följande citat: 
        \begin{quote}
          \blockcquote[s.\ 13]{Anderson2008sea}{%
            if an NSA employee is observed in a toilet stall at Baltimore 
            Washington airport selling key material to a Chinese diplomat, then 
            (assuming his operation was not authorized) we can describe him as 
            `trusted but not trustworthy'
          }.
        \end{quote}

      \item[Sekretess] Sekretess är en teknisk term för effekten av en mekanism 
        som begränsar antalet principals som kan ta del av information.

      \item[Konfidentialitet] Konfidentialitet syftar till att tillhandahålla 
        sekretess för andra principals hemliga information.

      \item[Personlig integritet] Detta är förmågan eller rätten att kunna 
        skydda sin personliga information.
        Det gäller alltså bara individer, exempelvis företag har ingen 
        personlig integritet.

      \item[Integritet] Detta är en teknisk term för egenskapen att data 
        förblir oförändrat, eller, om förändring sker ska den inte förbli 
        obemärkt.

      \item[Autenticitet] Detta begrepp innefattar integritet och fräshhet.
        Om kommunikation spelas in och sedan spelas upp vid ett annat 
        tillfälle, då kommer integriteten att ha bevarats men inte fräshheten 
        -- alltså är en återuppspelning inte autentisk.
    \end{description}
    Dessa definitioner stämmer även överens med RFC 4949~\cite{rfc4949}.
  \end{solution}


  
\question\label{q:usability:E:C:A}
  Human psychology is important in security.
  It is used in both security usability and social engineering.
  \begin{parts}
    \part[2] Give an overview of why psychology is important in 
    security.

    \begin{solution}
      Då systemen vi är beroende av och som ska upprätthålla vår säkerhet 
      handhas av människor blir psykologin genast viktig.
      Vi behöver psykologin inom säkerhetsområdet för att kunna ta hänsyn till 
      hur människor fungerar när vi konstruerar säkerhetssystem.
      Exempelvis, om vi gör ett system för komplext och användaren tycker att 
      komplexiteten är onödig, då kommer denne användare att aktivt försöka att 
      ta sig runt systemet -- kanske genom att skriva upp långa lösenord 
      istället för att lära sig dem utantill.

      Om vi däremot tar hänsyn till användarnas kognitiva begränsningar, då kan 
      vi konstruera system som både är säkra och enkla att använda.
    \end{solution}

    \part[4] Give an example of an attack which exploits weaknesses in human 
    psychology.
    Also explain why it works.

    \begin{solution}
      En psykologibaserad attack utnyttjar svagheter hos användarna för att ta 
      sig runt ett säkerhetssystem, det är alltså inte säkerhetssystemen som 
      angrips.

      Ett exempel på en sådan attack kan vara att en användare får ett e-brev 
      som till synes är från banken och som innehåller en länk till en 
      inloggningssida, kallat nätfiske.
      Brevet kan be användaren att uppdatera någonting hos banken via internet.
      Ett förfarande beskrivs och sedan läggs till \enquote{eller klicka på 
        länken}.
      Med en förfarande som låter som att det kan ta fem till tio klick kommer 
      användaren sannolikt att välja enklicksalternativet.
      Notera att förfarandet måste vara korrekt för banken medan länken är till 
      en phishingsida.
      Utformandet kan leda till vad litteraturen~\cite[s. 23]{Anderson2008sea} 
      kallar \emph{\foreignlanguage{english}{capture errors}}, att användaren 
      använder ett invant beteende: i detta fall att användaren klickar på 
      direktlänkar.

      Därutöver försöker nätfiskaren att få användaren att tillämpa fel regler 
      i situationen.
      Exempelvis, användaren kanske (omedvetet) lägger större vikt vid att ett 
      hänglås syns i webbläsaren för säker anslutning än att bankens namn är 
      rätt stavat i URL:en.
      Även att bankens namn finns med någonstans i URL:en kan vara en 
      tillräckligt stark regel för att användaren ska undvika att detektera den 
      felaktiga fiske-URL:en.
    \end{solution}
  \end{parts}


  
\end{questions}


\printbibliography{}
\end{document}
