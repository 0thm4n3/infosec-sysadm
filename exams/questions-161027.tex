\question\label{q:trustcomp:crypto:E:C}
  A user wishes to provide confidentiality to a file.
  \begin{parts}
    \part[3] She can accomplish this through mechanisms provided in the 
    operating system.
    Explain how this works and what are the limits.

    \part[3] She can also accomplish this through purely cryptographic 
    mechanisms.
    Explain how this works and what are the limits.
  \end{parts}

  \begin{solution}
    The first way she's securing her file is by using access control mechanisms 
    in the operating system (OS).

    Assuming we have physical access to the computer, then we can just read the 
    raw data from the disk.
    This can be accomplished by either booting our own OS on her computer, or 
    by removing the disk.

    If we don't have physical access we can always try to bypass the access 
    control mechanisms in other ways, e.g.\ by gaining privileges in the system 
    or seeing if the OS has failed to protect reading from the raw disk (i.e.\ 
    not using the file system).

    The main point here is that the operating system must be working correctly 
    for its mechanisms to be effective.
    The \emph{running} operating system will provide confidentiality by not 
    allowing other users' requests to open the file.

    The most obvious way to have system independent security for this file is 
    to encrypt it, i.e.~using cryptographic mechanisms.
    This way no one can read it unless they have access to the key, and this is 
    true no matter if you change the environment.
    (Of course, if the system is untrusted someone can get to the key that way, 
    but that's outside the scope of this question.)
  \end{solution}


  
\question\label{q:usability:E:C:A}
  Human psychology is important in security.
  It is used in both security usability and social engineering.
  \begin{parts}
    \part[2] Give an overview of why psychology is important in 
    security.

    \begin{solution}
      Då systemen vi är beroende av och som ska upprätthålla vår säkerhet 
      handhas av människor blir psykologin genast viktig.
      Vi behöver psykologin inom säkerhetsområdet för att kunna ta hänsyn till 
      hur människor fungerar när vi konstruerar säkerhetssystem.
      Exempelvis, om vi gör ett system för komplext och användaren tycker att 
      komplexiteten är onödig, då kommer denne användare att aktivt försöka att 
      ta sig runt systemet --- kanske genom att skriva upp långa lösenord 
      istället för att lära sig dem utantill.

      Om vi däremot tar hänsyn till användarnas kognitiva begränsningar, då kan 
      vi konstruera system som både är säkra och enkla att använda.
    \end{solution}

    \part[4] Give an example of an attack which exploits weaknesses in human 
    psychology.
    Also explain why it works.

    \begin{solution}
      En psykologibaserad attack utnyttjar svagheter hos användarna för att ta 
      sig runt ett säkerhetssystem, det är alltså inte säkerhetssystemen som 
      angrips.

      Ett exempel på en sådan attack kan vara att en användare får ett e-brev 
      som till synes är från banken och som innehåller en länk till en 
      inloggningssida, kallat nätfiske.
      Brevet kan be användaren att uppdatera någonting hos banken via internet.
      Ett förfarande beskrivs och sedan läggs till \enquote{eller klicka på 
        länken}.
      Med en förfarande som låter som att det kan ta fem till tio klick kommer 
      användaren sannolikt att välja enklicksalternativet.
      Notera att förfarandet måste vara korrekt för banken medan länken är till 
      en phishingsida.
      Utformandet kan leda till vad litteraturen~\cite[s. 23]{Anderson2008sea} 
      kallar \emph{\foreignlanguage{english}{capture errors}}, att användaren 
      använder ett invant beteende: i detta fall att användaren klickar på 
      direktlänkar.

      Därutöver försöker nätfiskaren att få användaren att tillämpa fel regler 
      i situationen.
      Exempelvis, användaren kanske (omedvetet) lägger större vikt vid att ett 
      hänglås syns i webbläsaren för säker anslutning än att bankens namn är 
      rätt stavat i URL:en.
      Även att bankens namn finns med någonstans i URL:en kan vara en 
      tillräckligt stark regel för att användaren ska undvika att detektera den 
      felaktiga fiske-URL:en.
    \end{solution}
  \end{parts}


  
\question[4]\label{q:sidechannels}
% examgen: sidechannels:E
Give an example of a side-channel attack and motivate why it is a side channel.

\begin{solution}
  A side channel is an unintended channel emitting information which is due 
  to physical implementation flaws and not theoretical weaknesses or forcing 
  attempts.

  (2 points) Extracting the secret key from a device by measuring energy 
  consumption or electromagnetic emissions while the device performs 
  computations using the secret key.

  (1 point) This is a side channel since it relies on a weakness in the 
  hardware implementation.
  (1 point) It is further an active attack since we might need the device to 
  perform operations on certain ciphertexts (or plaintexts).
\end{solution}


\question\label{q:foundations:E}
  Define the following terms:
  \begin{parts}
    \part[1] Trusted
    \part[1] Trustworthy
    \part[1] Secrecy
    \part[1] Confidentiality
    \part[1] Integrity
    \part[1] Authenticity
  \end{parts}

  \begin{solution}
    \citet[ss.\ 13--14]{Anderson2008sea} definierar begreppen enligt följande:
    \begin{description}
      \item[Pålitlighet] Ett system eller principal som innehar pålitlighet 
        (\foreignlanguage{english}{is trusted}) är ett system eller principal 
        som kan bryta din säkerhetspolicy.

      \item[Pålitlig] Ett system eller principal som är pålitlig 
        (\foreignlanguage{english}{is trustworthy}) är ett system eller 
        principal som inte kommer att misslyckas.
        (Den kommer alltså inte att bryta din säkerhetspolicy.)

        Ett exempel för att illustrera skillnaden ges av följande citat: 
        \blockcquote[s.\ 13]{Anderson2008sea}{%
          if an NSA employee is observed in a toilet stall at Baltimore 
          Washington airport selling key material to a Chinese diplomat, then 
          (assuming his operation was not authorized) we can describe him as 
          \enquote{trusted but not trustworthy}%
        }.

      \item[Sekretess] Sekretess är en teknisk term för effekten av en mekanism 
        som begränsar antalet principals som kan ta del av information.

      \item[Konfidentialitet] Konfidentialitet syftar till att tillhandahålla 
        sekretess för andra principals hemliga information.

      \item[Integritet] Detta är en teknisk term för egenskapen att data 
        förblir oförändrat, eller, om förändring sker ska den inte förbli 
        obemärkt.

      \item[Autenticitet] Detta begrepp innefattar integritet och fräshhet.
        Om kommunikation spelas in och sedan spelas upp vid ett annat 
        tillfälle, då kommer integriteten att ha bevarats men inte fräshheten 
        --- alltså är en återuppspelning inte autentisk.
    \end{description}
    Dessa definitioner stämmer även överens med RFC 4949~\cite{rfc4949}.
  \end{solution}


  
\question[3]\label{q:trustcomp}
% examgen: trustcomp:E:C:A
Describe the requirements for a process to be able to assess the integrity of 
itself and its execution environment.

\begin{solution}
  If the process can trust its environment (i.e.\ the operating system), then 
  it can rely on the environment to assess its own integrity.
  Thus the process relies on the integrity of the operating system.
  The oerating system in turn relies on the integrity of the hardware and must 
  rely on the hardware to assess its own integrity.
  Hence the process needs hardware that will not allow a modified version of 
  the operating system to run.
\end{solution}



\question[3]\label{q:accountability}
% tags: accountability:E:C
Explain the idea of double-entry book-keeping.

\begin{solution}
  It originates from banks.
  Every entry is either a credit or a debit.
  Every credit must have a corresponding debit, i.e.\ they cancel each other if 
  added together.
  This means that when all entries are added together, the final balance should 
  be zero.
  Thus, we keep the constant state of zero balance, and when the final balance 
  is non-zero we know that something is wrong.
\end{solution}


\question\label{q:auth:E:C}
  Describe the terms
  \begin{parts}
    \part[2] identification and
    \part[2] authentication.
  \end{parts}
  Make sure to illustrate your explanations by examples.
  You must also give an example of a mechanism for each of the terms.

  \begin{solution}
    In identification you claim an identity.
    This can be done using e.g.~a username, fingerprint or DNA sequence.

    In authentication you prove you are who you claim you are.
    This can be done using e.g.~\emph{who} you are (biometric), \emph{where} 
    you are (location) or what you \emph{do} (biometric), something you 
    \emph{have} (e.g.~BankID), or something you \emph{know} (password).
  \end{solution}


  
\question\label{q:infotheory:passwd:E:C}
  Explain how information theory can be used to estimate the strength of 
  passwords chosen under a given password composition policy:
  \begin{parts}
    \part[2] How can you estimate the upper bound, i.e.~the maximum possible 
    entropy?

    \part[2] Why can't you estimate any (useful) lower bound, i.e.~the minimum 
    possible entropy?

    \part[2] How can you estimate the average case, i.e.~what is usually the 
    case when users choose the passwords?
  \end{parts}

  \begin{solution}
    You assume that every part of the password is chosen uniformly randomly.
    This gives the maximum entropy, i.e.~it is an upper bound.
    You have to account for all choices the password composition policy allows.
    Or rather, all choices the policy removes.

    This is hard because a user can choose a very easy to guess password, which 
    has almost no entropy.
    Similarly, if all users choose the same password, then the entropy would be 
    zero.

    The average case can be estimated as in~\cite{Komanduri2011opa}.
    You have to \emph{sample a lot of user-generated passwords}, then you can 
    perform a frequency analysis to find the probabilities and compute the 
    entropy.
    The users are the stochastic variable (random output) and you must get 
    a large enough sample to estimate the probability distribution.
  \end{solution}


  
\question[3]\label{q:software}
% tags: software:A
Can a files such as images (e.g.\ JPEGs) and other data be dangerous?
(Do not answer yes or no, answer why or why not.)

\begin{solution}
  Yes, they can contain machine code which can be executed if there is e.g.\ 
  a buffer overrun vulnerability in the software that reads the data.
\end{solution}


